\PassOptionsToPackage{unicode=true}{hyperref} % options for packages loaded elsewhere
\PassOptionsToPackage{hyphens}{url}
%
\documentclass[]{book}
\usepackage{lmodern}
\usepackage{amssymb,amsmath}
\usepackage{ifxetex,ifluatex}
\usepackage{fixltx2e} % provides \textsubscript
\ifnum 0\ifxetex 1\fi\ifluatex 1\fi=0 % if pdftex
  \usepackage[T1]{fontenc}
  \usepackage[utf8]{inputenc}
  \usepackage{textcomp} % provides euro and other symbols
\else % if luatex or xelatex
  \usepackage{unicode-math}
  \defaultfontfeatures{Ligatures=TeX,Scale=MatchLowercase}
\fi
% use upquote if available, for straight quotes in verbatim environments
\IfFileExists{upquote.sty}{\usepackage{upquote}}{}
% use microtype if available
\IfFileExists{microtype.sty}{%
\usepackage[]{microtype}
\UseMicrotypeSet[protrusion]{basicmath} % disable protrusion for tt fonts
}{}
\IfFileExists{parskip.sty}{%
\usepackage{parskip}
}{% else
\setlength{\parindent}{0pt}
\setlength{\parskip}{6pt plus 2pt minus 1pt}
}
\usepackage{hyperref}
\hypersetup{
            pdftitle={CM 2110 Calculus and Statistical Distributions},
            pdfauthor={Dr.~Priyanga D. Talagala},
            pdfborder={0 0 0},
            breaklinks=true}
\urlstyle{same}  % don't use monospace font for urls
\usepackage{longtable,booktabs}
% Fix footnotes in tables (requires footnote package)
\IfFileExists{footnote.sty}{\usepackage{footnote}\makesavenoteenv{longtable}}{}
\usepackage{graphicx,grffile}
\makeatletter
\def\maxwidth{\ifdim\Gin@nat@width>\linewidth\linewidth\else\Gin@nat@width\fi}
\def\maxheight{\ifdim\Gin@nat@height>\textheight\textheight\else\Gin@nat@height\fi}
\makeatother
% Scale images if necessary, so that they will not overflow the page
% margins by default, and it is still possible to overwrite the defaults
% using explicit options in \includegraphics[width, height, ...]{}
\setkeys{Gin}{width=\maxwidth,height=\maxheight,keepaspectratio}
\setlength{\emergencystretch}{3em}  % prevent overfull lines
\providecommand{\tightlist}{%
  \setlength{\itemsep}{0pt}\setlength{\parskip}{0pt}}
\setcounter{secnumdepth}{5}
% Redefines (sub)paragraphs to behave more like sections
\ifx\paragraph\undefined\else
\let\oldparagraph\paragraph
\renewcommand{\paragraph}[1]{\oldparagraph{#1}\mbox{}}
\fi
\ifx\subparagraph\undefined\else
\let\oldsubparagraph\subparagraph
\renewcommand{\subparagraph}[1]{\oldsubparagraph{#1}\mbox{}}
\fi

% set default figure placement to htbp
\makeatletter
\def\fps@figure{htbp}
\makeatother

\usepackage{booktabs}
\usepackage{amsthm}
\makeatletter
\def\thm@space@setup{%
  \thm@preskip=8pt plus 2pt minus 4pt
  \thm@postskip=\thm@preskip
}
\makeatother
\usepackage{fancyhdr}
\pagestyle{fancy}
\fancyfoot[CO,CE]{Prepared by Dr. Priyanga D. Talagala}
\fancyfoot[LE,RO]{\thepage}
\usepackage{wrapfig}
\usepackage{floatrow}
\floatplacement{figure}{H}
\floatplacement{table}{H}
\makeatletter\renewcommand*{\fps@figure}{H}\makeatother
\usepackage{mathtools}
\usepackage[]{natbib}
\bibliographystyle{apalike}

\title{CM 2110 Calculus and Statistical Distributions}
\author{Dr.~Priyanga D. Talagala}
\date{2020-05-28}

\begin{document}
\maketitle

{
\setcounter{tocdepth}{1}
\tableofcontents
}
\hypertarget{course-syllabus}{%
\chapter*{Course Syllabus}\label{course-syllabus}}
\addcontentsline{toc}{chapter}{Course Syllabus}

\hypertarget{pre-requisites}{%
\section*{Pre-requisites}\label{pre-requisites}}
\addcontentsline{toc}{section}{Pre-requisites}

CM 1110

\textbf{Remark:}

\emph{This course module contains two main sections: (1) mathematics and (2) statistics. This syllabus is designed for the statistics section. Lectures for mathematics section and statistics section are conducted by two lecturers as two separate sub modules (1.5 hour lectures/Week). End Semester Examination is conducted as a single examination.}

\hypertarget{learning-outcomes}{%
\section*{Learning Outcomes}\label{learning-outcomes}}
\addcontentsline{toc}{section}{Learning Outcomes}

On successful completion of this module, students will be able to plan more carefully the design of experiment in advance which provide evidence for or against theories of cause and effect and make inferences about population characteristics based on sample information and thereby solve data analysis problems in different application domains. (R(\url{https://cran.r-project.org/}) and RStudio are also freely available to install on your own computer). Get the Open Source Edition of RStudio Desktop. RStudio allows you to run R in a more user-friendly environment.

\hypertarget{outline-syllabus}{%
\section*{Outline Syllabus}\label{outline-syllabus}}
\addcontentsline{toc}{section}{Outline Syllabus}

\begin{itemize}
\tightlist
\item
  Functions of Several Variables
\item
  Linear Algebra
\item
  Coordinate Systems \& Vectors
\item
  Differential Equations
\item
  \textbf{Statistical Distributions}
\item
  \textbf{Estimation}
\item
  \textbf{Hypothesis Testing}
\item
  \textbf{Design of Experiments}
\end{itemize}

\hypertarget{method-of-assessment}{%
\section*{Method of Assessment}\label{method-of-assessment}}
\addcontentsline{toc}{section}{Method of Assessment}

\begin{itemize}
\tightlist
\item
  Mid-semester examination
\item
  End-semester examination
\end{itemize}

\hypertarget{recommended-texts}{%
\section*{Recommended Texts}\label{recommended-texts}}
\addcontentsline{toc}{section}{Recommended Texts}

\begin{itemize}
\tightlist
\item
  Casella, G., \& Berger, R. L. (2002). Statistical inference (Vol. 2, pp.~337-472). Pacific Grove, CA: Duxbury.
\item
  Mood, A.M., Graybill, F.A. and Boes, D.C. (2007): Introduction to the Theory of Statistics, 3rd Edn.
  (Reprint). Tata McGraw-Hill Pub. Co.~Ltd.~
\item
  Montgomery, D. C. (2017). Design and analysis of experiments. John wiley \& sons.
\end{itemize}

\hypertarget{lecturer}{%
\section*{Lecturer}\label{lecturer}}
\addcontentsline{toc}{section}{Lecturer}

Dr.~Priyanga D. Talagala

\hypertarget{schedule}{%
\section*{Schedule}\label{schedule}}
\addcontentsline{toc}{section}{Schedule}

Lectures:

\begin{itemize}
\tightlist
\item
  Friday {[}9.15 am - 10.45 am{]}
\end{itemize}

Tutorial:

\begin{itemize}
\tightlist
\item
  Friday {[}11.00 am - 12.30 pm{]}
\end{itemize}

Consultation time:

\begin{itemize}
\tightlist
\item
  Friday {[}8.15 am to 9.00 am{]}
\end{itemize}

\hypertarget{statistical-distributions}{%
\chapter{Statistical Distributions}\label{statistical-distributions}}

\pagenumbering{arabic}

\hypertarget{random-variable}{%
\section{Random Variable}\label{random-variable}}

\begin{itemize}
\tightlist
\item
  Some sample spaces contain quantitative (numerical) outcomes, others contain qualitative outcomes.
\item
  Often it is convenient to work with sample spaces containing numerical outcomes.
\item
  A function that maps the original sample space into the real numbers is called a `random variable'.
\item
  This is more useful when the original sample space contains qualitative outcomes.
\end{itemize}

\textbf{Definition 1: Random Variable}

A \textbf{random variable} is a function from a sample space \(S\) into the real numbers (\emph{i.e.} \(X:S \rightarrow \Re\))

\begin{center}\includegraphics[width=0.5\linewidth]{figure/Ch1_F1} \end{center}

\begin{itemize}
\tightlist
\item
  In other words, to each one of the outcomes of an experiment or a sample point \(\omega_i\), of the sample spaces, there is a unique real number \(x_i\), known as the value of the random variable \(X\).
\item
  The range of the random variable is called the \emph{induced sample space}.
\item
  \emph{A note on notation:} Random variables will always denoted with uppercase letters and the realized values of the random variable (or its range) will be denoted by the corresponding lowercase letters. Thus, the random variable \(X\) can take the value \(x\).
\end{itemize}

\hypertarget{types-of-random-variables}{%
\subsection{Types of Random Variables}\label{types-of-random-variables}}

\begin{itemize}
\tightlist
\item
  A random variable is of two types

  \begin{itemize}
  \tightlist
  \item
    Discrete Random Variable
  \item
    Continuous Random Variable
  \end{itemize}
\end{itemize}

\hypertarget{discrete-random-variable}{%
\subsubsection{Discrete Random Variable}\label{discrete-random-variable}}

\begin{itemize}
\tightlist
\item
  If the induced sample space is discrete, then the random variable is called a \textbf{discrete random variable}.
\item
  A random variable (\(X\)) is said to be discrete if it takes only a finite; or an infinite but countable number of values.
\item
  Examples: The following are discrete random variables:

  \begin{itemize}
  \tightlist
  \item
    Number of children per family
  \item
    Attendance of CM 2110 lectures
  \item
    GPA credit value that you can obtain for CM 2110
  \item
    The number of machine breakdowns during a given day in a company
  \end{itemize}
\end{itemize}

\emph{Example 01}
Consider the experiment of tossing a coin

\begin{center}\includegraphics[width=1\linewidth]{figure/Ch1box1-1} \end{center}

\emph{Example 02}

Consider the experiment of rolling of a single dice

\begin{center}\includegraphics[width=1\linewidth]{figure/Ch1box2-1} \end{center}

\emph{Example 03}

Consider the experiment of tossing two coins and count the number of heads turn up head

\begin{center}\includegraphics[width=1\linewidth]{figure/Ch1box3-1} \end{center}

\hypertarget{continuous-random-variable}{%
\subsubsection{Continuous Random Variable}\label{continuous-random-variable}}

\begin{itemize}
\tightlist
\item
  If the induced sample space is continuous, then the random variable is called a \textbf{continuous random variable.}
\item
  A continuous random variable is a random variable that can take on any value in a given interval.
\item
  Random variables which consist of measurements are usually continuous.
\item
  For example

  \begin{itemize}
  \tightlist
  \item
    height of a student in this class
  \item
    the current measured in a thin copper wire in milliamperes
  \item
    Life time of a mobile phone battery
  \item
    SGPA of a level 2 student
  \end{itemize}
\end{itemize}

\emph{Example 04}

Lifetime of a bulb

\begin{center}\includegraphics[width=1\linewidth]{figure/Ch1box4-1} \end{center}

\hypertarget{probability-mass-function}{%
\section{Probability Mass Function}\label{probability-mass-function}}

\textbf{Definition 2: Discrete density function of a discrete random variable}

If \(X\) is a discrete random variable with distinct values \(x_1, x_2, \dots, x_n, \dots,\) then the function, denoted by \(f_X(.)\) and defined by

\begin{equation}
f_X(x) =
\begin{cases} 
P(X=x) & \text{if } x=x_j, j=1,2,\dots,n,\dots\\
0 & \text{if } x \neq x_j
\end{cases}
\end{equation}

is defined to be the discrete density function of \(X\).

\begin{itemize}
\tightlist
\item
  The values of a discrete random variable are often called \emph{mass points.}
\item
  \(f_X(x)\) denotes the \emph{mass} associated with the \emph{mass point} \(x_j\).
\item
  \textbf{\emph{Probability mass function}} \emph{discrete frequency function} and \emph{probability function} are other terms used in place of \emph{discrete density function}
\item
  Probability function gives the measure of probability for different values of \(X\).
\end{itemize}

\hypertarget{properties-of-a-probability-mass-function}{%
\subsection{Properties of a Probability Mass Function}\label{properties-of-a-probability-mass-function}}

\begin{itemize}
\tightlist
\item
  Let \(X\) be a discrete random variable with probability mass function \(f_X(x)\). Then,
\end{itemize}

\begin{enumerate}
\def\labelenumi{\arabic{enumi}.}
\tightlist
\item
  For any \(x\in \Re\), \(0\leq f_X(x) \leq 1.\)
\item
  Let \(E\) be an event and \(I= \{X(\omega):\omega \in E\}.\) Then \(P(E) = P(X\in I) = \sum_{x \in I}f_X(x).\)
\item
  Let \(R = \{X(\omega):\omega \in \Omega\}.\) Then \(\sum_{x\in \Re} f_X(x) = 1.\)
\end{enumerate}

\hypertarget{representations-of-probability-mass-functions}{%
\subsection{Representations of Probability Mass Functions}\label{representations-of-probability-mass-functions}}

\hypertarget{using-a-table}{%
\subsubsection{Using a table}\label{using-a-table}}

\begin{center}\includegraphics[width=1\linewidth]{figure/Ch1box5-1} \end{center}

\hypertarget{using-a-function}{%
\subsubsection{Using a function}\label{using-a-function}}

\begin{center}\includegraphics[width=1\linewidth]{figure/Ch1box6-1} \end{center}

\hypertarget{using-a-graph}{%
\subsubsection{Using a graph}\label{using-a-graph}}

\begin{center}\includegraphics[width=1\linewidth]{figure/Ch1box7-1} \end{center}

\hypertarget{probability-density-function}{%
\section{Probability density function}\label{probability-density-function}}

\hypertarget{cumulative-distribution-function}{%
\section{Cumulative distribution function}\label{cumulative-distribution-function}}

\hypertarget{descriptive-properties-of-distributions}{%
\section{Descriptive properties of distributions}\label{descriptive-properties-of-distributions}}

\hypertarget{models-for-discrete-distributions}{%
\section{Models for discrete distributions}\label{models-for-discrete-distributions}}

\hypertarget{models-for-continuous-distributions}{%
\section{Models for continuous distributions}\label{models-for-continuous-distributions}}

\hypertarget{estimations}{%
\chapter{Estimations}\label{estimations}}

\hypertarget{point-estimation}{%
\section{Point Estimation}\label{point-estimation}}

\hypertarget{methods-of-finding-point-estimators}{%
\subsection{Methods of finding point estimators}\label{methods-of-finding-point-estimators}}

\hypertarget{methods-of-evaluating-point-estimators}{%
\subsection{Methods of evaluating point estimators}\label{methods-of-evaluating-point-estimators}}

\hypertarget{interval-estimation}{%
\section{Interval Estimation}\label{interval-estimation}}

\hypertarget{interpretation-of-confidence-intervals}{%
\subsection{Interpretation of confidence intervals}\label{interpretation-of-confidence-intervals}}

\hypertarget{methods-of-finding-interval-estimators}{%
\subsection{Methods of finding interval estimators}\label{methods-of-finding-interval-estimators}}

\hypertarget{methods-of-evaluating-interval-estimators}{%
\subsection{Methods of evaluating interval estimators}\label{methods-of-evaluating-interval-estimators}}

\hypertarget{hypothesis-testing}{%
\chapter{Hypothesis Testing}\label{hypothesis-testing}}

\hypertarget{null-and-alternative-hypotheses}{%
\section{Null and alternative hypotheses}\label{null-and-alternative-hypotheses}}

\hypertarget{errors-in-testing-hypotheses-type-i-and-type-ii-error}{%
\section{Errors in testing hypotheses-type I and type II error}\label{errors-in-testing-hypotheses-type-i-and-type-ii-error}}

\hypertarget{significance-level-size-power-of-a-test}{%
\section{Significance level, size, power of a test}\label{significance-level-size-power-of-a-test}}

\hypertarget{formulation-of-hypotheses}{%
\section{Formulation of hypotheses}\label{formulation-of-hypotheses}}

\hypertarget{methods-of-testing-hypotheses}{%
\section{Methods of testing hypotheses}\label{methods-of-testing-hypotheses}}

\hypertarget{design-of-experiments}{%
\chapter{Design of Experiments}\label{design-of-experiments}}

\hypertarget{ntroduction-to-experimental-design}{%
\section{ntroduction to experimental design}\label{ntroduction-to-experimental-design}}

\hypertarget{basic-principles-of-experimental-design}{%
\section{Basic principles of experimental design}\label{basic-principles-of-experimental-design}}

\hypertarget{completely-randomized-design}{%
\section{Completely randomized design}\label{completely-randomized-design}}

\bibliography{book.bib,packages.bib}

\end{document}
